\documentclass[]{article}

%opening
\title{Using Quasi-static Magnetic Fields and Meta-material Amplifiers for ubiquitous wireless charging}
\author{Jordan Hill}

\begin{document}

\maketitle

\begin{abstract}
	
\end{abstract}

\section{Introduction}
Wireless Charging, the Pinnacle of refueling devices, free of constraints, nothing holding the user back from being forced to charge with the length of cord limiting there freedom to move around while charging there device. But, a problem is present with Modern Wireless Charging in the current form that it is used in currently, and that is the need for a connection constantly to charge your device. The current method of wireless charging used today is known as Qi Charging. Qi Charging works with the wireless charging device having a built in copper coil. The Copper Coil is attached to the battery of the device, so current is put into the battery correctly while the device is charging. 
\section{Background}

\section{Math Modeling}

\section{Deriving of FDTD Equations}

\section{Quasi-static Magnetic Field Equations}

\section{Metamaterial Equations} 

\section{S-Parameter Matrices}

\section{COSMOL Simulations}



\end{document}
