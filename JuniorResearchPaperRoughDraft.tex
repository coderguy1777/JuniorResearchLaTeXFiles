\documentclass[]{article}

%opening
\title{Using Quasi-static Magnetic Fields and Meta-material Amplifiers to allow for Ubiquitous Wireless Charging}
\author{Jordan Hill \\ David Stover}

\begin{document}
	
\maketitle

\pagebreak

\begin{abstract}
Quasi-static Magnetic fields are Electromagnetic fields that are designed that have current direction, $\alpha$, pass through a series of discrete capacitors, which can be defined as $\mu_{x}$, and so on depending on the number of capacitors being used to generate the Quasi-static Magnetic Field. This, combined with the use of Meta-materials, that have the ability to allow for high amounts of electric resonance, have the possibility to allow for highly efficient wireless charging, and solve the problem of current wireless charging, and it being restricted to the device always being stuck to a pad while charging, and not allowing the user the use of mobility while using wireless charging with modern methods. Hypothetically, to make the process of wireless charging hundreds of time more efficient through use of Quasi-static magnetic fields and reflective meta-materials 
\end{abstract}

\pagebreak


Wireless Charging, the Pinnacle of refueling devices, free of constraints, nothing holding the user back from being forced to charge with the length of cord limiting there freedom to move around while charging there device. But, a problem is present with Modern Wireless Charging in the current form that it is used in currently, and that is the need for a connection constantly to charge your device. The current method of wireless charging used today is known as Qi Charging. Qi Charging works with the wireless charging device having a built in copper coil. The Copper Coil is attached to the battery of the device, so current is put into the battery correctly while the device is charging. Power is transferred to the device through a Wireless Charging Pad, containing a Copper Coil inside the Pad. Transfer of Power is done between these two coils, using the coil inside the device as an inducting coil, and an alternating electromagnetic field is created. This field is the process responsible for allowing the wireless charging to occur in Qi Charging, and charge the device in Question. But, there is a problem with this process, and that is the lack of mobility, as when you place the device on the pad, you cannot under any circumstances remove it, as if you remove it from the pad, you lose the electromagnetic field that was made between the two coils, one in the pad and the device itself, which stops the charging process itself, which shows the Problem of Mobility with this method of wireless charging comes into play. Now, the answer to this problem is quite simple, and that is with the use of a Material, a special material, and the types of that being referred to as Meta-materials. Meta-materials are special materials, designed to have properties not found in any other modern material that would be found in Nature, and have a variety of properties when produced. Some of these materials have been known to have the ability to reflect electromagnetic waves, and reflect them completely, having no interaction with the Material itself what so ever. This application of such Meta-materials has great application to solve the problem of mobility with wireless charging. Meta-materials can contain these properties, mainly due to the construction of there unit cells, which make up the fundamental structure of the material itself, and its properties, such as impedance, admittance, and reflective properties of the material. Impedance is the effect of electrical resistance on an electrical component when compared to the alternating current, if one is present. The Reflective properties of the material mainly if the material doesn't absorb any of the Electromagnetic waves from the Quasistatic Magnetic field, but reflects them instead rather than absorbing the Electromagnetic waves from the Quasi-static magnetic field.
\section{Methods and Materials} 
Materials used in this project were quite minimal, due to the centralized nature of it heavily around simulations of the project, before any theoretical modeling was actually done.
\subsection{Mathematical Modeling}
Mathematical Modeling used within this project worked with many equations, but one of the main ones being for the modeling of the Quasi-static field coming from the main Quasi-static magnetic field generator itself. To define this "Generator" of the Quasi-static magnetic field, a Cylinder was designed for the theoretical model, which is for the cylinder, the length being symbolized by it being approximately equal to $\Delta$l, as shown here: $\Delta$l $\approx$ 2.4384 m. Visualizing this 
\subsection{Design of Meta-material Unit Cell}
When designing the Meta-material Unit Cell, multiple factors were taken into account for the design, that being the greatest amount of resonance that would have to be achieved. 
\subsection{COSMOL Simulations}
For the simulations for this project within COSMOL, that being the version of COSMOL being 5.2A as the version of COSMOL used for the simulation part of this project. The first simulation done was simulating the Quasi-static magnetic field generator, and the field itself when generated. For this simulation, the main generator, here will be defined as $\phi_c$, with the purpose of the c subscript defining the center generator and Quasi-static magnetic field center point. Then a cylinder was created with the length parameters for the height of the generator being defined as $\Delta$l $\approx$ 2.4384 m, and the radius of the generator being defined as $\Delta$r = 0.75 m. The Cylinder was then split along the center to define two separate domains to add simulation 

\section{Data Collection}
Collection of data was done mainly with the harvesting of data from COSMOL, and taking the data that obtained from the COSMOL Simulations. These data-sets were about 5 sets of data in total, but totaling for the lowest data-set with about 300 points of data within that smallest data-set, and nearly 14000 in other data-sets as well.

\section{Results}
Results from my data were mainly 
\section{Discussion}

\section{Conclusion}

\section{Acknowledgments}
Relating to any Acknowledgments that I have for anyone for making my project possible and for it succeeding mainly,  
\section{Appendix}

\section{Sources}

\end{document}