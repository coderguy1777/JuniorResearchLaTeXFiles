\documentclass[]{article}

%opening
\title{Using Quasi-static Magnetic Fields and Meta-material Amplifiers to allow for ubiquitous Wireless Charging}
\author{Jordan Hill}
\begin{document}

\maketitle

\begin{abstract}
Quasi-static Magnetic fields are Electromagnetic fields that are designed to have the direction of the current pass through a series of discrete 
\end{abstract}

\section{Introduction}
Wireless Charging, the Pinnacle of refueling devices, free of constraints, nothing holding the user back from being forced to charge with the length of cord limiting there freedom to move around while charging there device. But, a problem is present with Modern Wireless Charging in the current form that it is used in currently, and that is the need for a connection constantly to charge your device. The current method of wireless charging used today is known as Qi Charging. Qi Charging works with the wireless charging device having a built in copper coil. The Copper Coil is attached to the battery of the device, so current is put into the battery correctly while the device is charging. Power is transferred to the device through a Wireless Charging Pad, containing a Copper Coil inside the Pad. Transfer of Power is done between these two coils, using the coil inside the device as an inducting coil, and an alternating electromagnetic field is created. This field is the process responsible for allowing the wireless charging to occur in Qi Charging, and charge the device in Question. But, there is a problem with this process, and that is the lack of mobility, as when you place the device on the pad, you cannot under any circumstances remove it, as if you remove it from the pad, you lose the electromagnetic field that was made between the two coils, one in the pad and the device itself, which stops the charging process itself, which shows the Problem of Mobility with this method of wireless charging comes into play. Now, the answer to this problem is quite simple, and that is with the use of a Material, a Special material, and the types of that being referred to as Meta-materials. Meta-materials are special materials, designed to have properties not found in any other modern material that would be found in Nature, and have a variety of properties when produced. Some of these materials have been known to have the ability to reflect electromagnetic waves, and reflect them completely, having no interaction with the Material itself what so ever. This application of such Meta-materials has great application to solve the problem of mobility with wireless charging. Meta-materials can contain these properties, mainly due to the construction of there unit cells, which make up the fundamental structure of the material itself, and its properties, such as impedance within the material. Impedance  
\section{Background Research}

\section{Mathematical Modeling}

\section{COSMOL Simulations}

\end{document}